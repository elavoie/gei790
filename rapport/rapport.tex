\documentclass[12pt,english,frenchb,letterpaper]{article}
\usepackage[T1]{fontenc}
\usepackage[utf8]{inputenc}
\usepackage{geometry}
\usepackage{setspace}
\usepackage{graphics}
\usepackage{graphicx}
\usepackage{amsmath,amssymb,amsfonts,textcomp}
\usepackage{color}
\usepackage{calc}
\usepackage{verbatim}
\usepackage{longtable}
\geometry{verbose, letterpaper,tmargin=3cm,bmargin=3cm,lmargin=3cm,rmargin=3cm}
\usepackage[french]{babel}


% Custom commands for the report
\newcommand{\action}[3] {
\begin{tabbing}
\it{Action}\=\it{(#1)},\\
\> PRECOND: 
     #2 
     \\
\> EFFECT: \it{#3} .\\
\end{tabbing}}



\begin{document}
\thispagestyle{empty}
\begin{center}
{\large\em GEI790:  Intelligence Artificielle formelle}
\vfill
Rapport de l'App 1\\
présenté professorale de S8\\
\vfill
par:\\
Alexandre Malo\hspace{2cm} 05 659 076\\
Erick Lavoie\hspace{2cm} 05 646 456\\
\vfill
Date: \today
\end{center}

\newpage
\onehalfspacing


\tableofcontents

\newpage


\section{Introduction}
% Alex
% 0.5 page

\section{Recommandation et justification de la technique de planification}
% Erick
% 3 pages

\section{Resultats de tests}
% Alex et Erick
% 3 pages

% Criteres
% - Temps de calcul
% - Utilisation memoire
% - Qualite du plan
% - Complexite de developement

% - Couvrir les algorithmes de recherche informee et non-informee

\section{Representation STRIPS ou ADL des actions possibles de BlockBlitz}
% Erick 
% 1.5 pages
Par soucis de concision et de lisibilité, l'environnement de jeu initial et l'environnement de jeu final seront omis des paramètres, des préconditions et des effets des actions présentées bien qu'ils soient utilisés de fa\c con explicite dans le programme remis.  Pour l'ensemble des actions présentées, nous assumerons un environnement de jeu composé des faits suivants, les multiplicités présentées représentent le nombre de faits distincts du type présenté pouvant être retrouvés dans l'environnement, ex: nbJoueur(N), 1 signifie qu'un seul fait de type nbJoueur est présent dans l'environnement et block(...), 1+ signifie que chaque bloc est représenté par son propre fait et qu'il y en a autant que de blocs.

\begin{center}
  \begin{tabular}{@{} ccc @{}}
    \hline
    Fait & Multiplicité & Signification \\ 
    \hline
    NbJoueur(n) & 1 & $n$ est le nombre de joueurs \\ 
    NbBlock(n) & 1 & $n$ est le nombre de blocks \\ 
    NbColonnes(n) & 1 & $n$ est le nombre de colonnes \\ 
    NbRangees(n) & 1 & $n$ est le nombre de rangées \\ 
    Block(id,x,y) & 1+ & $id$ est la valeur du block, \\
                                  && $x$ est la position en selon $x$ \\
                                  && et $y$ est la position selon $y$ \\ 
    Player(id,nom,x,y,blockId) & 1+ & $id$ est l'identifiant unique du joueur \\
                                                              && $nom$ est le nom assigné au joueur \\ 
                                                              && $x$ est la position selon $x$ \\
                                                              && $y$ est la position selon $y$ \\
                                                              && $blockId$ est l'identifiant du bloc possédé par le joueur\\
    Nom(nom) & 1 & $nom$ est le nom du joueur \\ 
    \hline
  \end{tabular}
\end{center}

En plus de ces faits de base, les prédicats suivants sont utilisés pour faciliter l'expression des préconditions et des effets. \\
\\
Il est nécessaire de traduire la direction en coordonnées. Pour ce faire, le prédicat $Direction(posx,posy,newx,newy,direction)$ est utilisé pour déterminer la nouvelle position du joueur à partir de sa position précédente.  Il y a huit entrées pour chacune des direction.  L'entrée pour la direction $1$ est la suivante, les autres entrées sont laissées comme exercice au lecteur.
\begin{equation*}
	Direction(posx,posy,newx,newy, 1) \Leftarrow newy = posy + 1 \wedge newx = posx 
\end{equation*}

Les deux derniers prédicats sont les suivants.  $Empty(posx,posy)$ assure que la case visée ne contient rien, c'est-à-dire aucun joueur ou bloc, et $Exist(posx,posy)$ valide que la case visée appartient bel et bien à l'environnement, c'est-à-dire qu'elle est contenue dans ses limites.
\begin{eqnarray*}
	Empty(posx,posy) \Leftarrow & Block(\_,posx,posy)  \wedge Player(\_,\_,posx,posy,\_) \\
	Exist(posx,posy) \Leftarrow & posx  \geq 0 \wedge posy \geq 0 \\
	                                                    \wedge & NbColonnes(nbColonnes)  \wedge NbRangees(nbRangees) \\
	                                                   \wedge & posx < nbColonnes \wedge posy < nbRangees 
\end{eqnarray*}

Il est alors très aisé d'exprimer les préconditions et les effets de chacune des actions.

\begin{tabbing}
\it{Action}\=\it{(Move(direction))}\=,\\
\> PRECOND: \> $Nom(nom) \wedge Player(id,nom,posx,posy,blockid)$ \\ 
\> \> $  \wedge \ Direction(posx,posy,newx,newy,direction) $ \\
\> \> $   \wedge \ Exist(newx,newy) \wedge Empty(newx,newy) $ \\               
\> EFFECT: \>$ \lnot Player(id,nom,posx,posy,blockid) $ \\
\>  \> $\wedge\ Player(id,nom,newx,newy,blockid)$ .\\
\end{tabbing}

Pour le cas où le joueur ne possède pas de bloc, l'action $take$ est la suivante:

\begin{tabbing}
\it{Action}\=\it{(Take(direction))}\=,\\
\> PRECOND: \> $Nom(nom) \wedge Player(id,nom,posx,posy,0)$ \\ 
\> \> $  \wedge \ Direction(posx,posy,newx,newy,direction) $ \\
\> \> $   \wedge \ Exist(newx,newy) \wedge Block(newblockid,newx,newy) $ \\               
\> EFFECT: \>$ \lnot Player(id,nom,posx,posy,0) $ \\
\>  \> $\wedge\ Player(id,nom,newx,newy,newblockid)$ \\
\> \> $\wedge\ \lnot Block(newblockid,newx,newy) $ .\\
\end{tabbing}

Pour le cas où le joueur possède déjà un bloc, l'action $take$ est plutôt:
\begin{tabbing}
\it{Action}\=\it{(Take(direction))}\=,\\
\> PRECOND: \> $Nom(nom) \wedge Player(id,nom,posx,posy,blockid)$ \\ 
\> \> $  \wedge \ Direction(posx,posy,newx,newy,direction) $ \\
\> \> $   \wedge \ Exist(newx,newy) \wedge Block(newblockid,newx,newy) $ \\               
\> EFFECT: \>$ \lnot Player(id,nom,posx,posy,blockid) $ \\
\>  \> $\wedge\ Player(id,nom,newx,newy,newblockid)$ \\
\> \> $\wedge\ \lnot Block(newblockid,newx,newy) $ \\
\> \> $\wedge\ Block(blockid,newx,newy) $ .\\
\end{tabbing}

\begin{tabbing}
\it{Action}\=\it{(Drop(direction))}\=,\\
\> PRECOND: \> $Nom(nom) \wedge Player(id,nom,posx,posy,blockid)$ \\ 
\> \> $  \wedge \  \lnot Player(id,nom,posx,posy,0) $ \\
\> \> $  \wedge \ Direction(posx,posy,newx,newy,direction) $ \\
\> \> $   \wedge \ Exist(newx,newy) \wedge Empty(newx,newy) $ \\               
\> EFFECT: \>$ \lnot Player(id,nom,posx,posy,blockid) $ \\
\>  \> $\wedge\ Player(id,nom,posx,posy,0)$ \\
\> \> $\wedge\ Block(blockid,newx,newy) $ .\\
\end{tabbing}

Bien que la problématique présente deux cas distincts pour l'attaque, avec la représentation proposée, la même description s'applique dans les deux cas (joueur possède ou non un bloc).  De plus, pour fins de planification, nous allons considérer que l'action réussi toujours.:

\begin{tabbing}
\it{Action}\=\it{(Attack(direction))}\=,\\
\> PRECOND: \> $Nom(nom) \wedge Player(id,nom,posx,posy,blockid)$ \\ 
\> \> $  \wedge \ Direction(posx,posy,newx,newy,direction) $ \\
\> \> $  \wedge \ Player(id2,nom2,newx,newy,blockid2) $ \\
\> \> $  \wedge \ \lnot Player(id2,nom2,newx,newy,0) $ \\
\> \> $ \wedge\ \lnot Block(bid,newx,newy) \wedge \ Exist(newx,newy)$ \\               
\> EFFECT: \>$ \lnot Player(id,nom,posx,posy,blockid) $ \\
\>  \> $\wedge\ Player(id,nom,posx,posy,blockid2)$ \\
\>  \> $\wedge\ \lnot Player(id2,nom2,newx,newy,blockid2)$ \\
\>  \> $\wedge\ Player(id2,nom2,newx,newy,blockid)$ .\\
\end{tabbing}

Finalement, l'action $None$ ne possède aucune précondition et n'a aucun effet sur l'environnement.

\section{Description du vocabulaire de nos parties principales }
% En logique du premier ordre
% Erick 
% 1 page

\section{Description de parties principales de notre solution}
% Alex
% 1 page

\section{Recommandation de la pertinence des methodes d'algorithme évolué}
% Alex
% 1 page

\section{Conclusion}
% Alex
% 0.5 page

\end{document}