\documentclass[12pt,english,frenchb,letterpaper]{article}
\usepackage[T1]{fontenc}
\usepackage[utf8]{inputenc}
\usepackage{geometry}
\usepackage{setspace}
\usepackage{graphics}
\usepackage{graphicx}
\usepackage{amsmath,amssymb,amsfonts,textcomp}
\usepackage{color}
\usepackage{calc}
\usepackage{verbatim}
\usepackage{longtable}
\geometry{verbose, letterpaper,tmargin=3cm,bmargin=3cm,lmargin=3cm,rmargin=3cm}
\usepackage[french]{babel}


% Custom commands for the report
\newcommand{\action}[3] {
\begin{tabbing}
\it{Action}\=\it{(#1)},\\
\> PRECOND: 
     #2 
     \\
\> EFFECT: \it{#3} .\\
\end{tabbing}}



\begin{document}
\thispagestyle{empty}
\begin{center}
{\large\em GEI790:  Intelligence Artificielle formelle}
\vfill
Rapport de l'App 1\\
présenté professorale de S8\\
\vfill
par:\\
Alexandre Malo\hspace{2cm} 05 659 076\\
Erick Lavoie\hspace{2cm} 05 646 456\\
\vfill
Date: \today
\end{center}

\newpage
\onehalfspacing


\tableofcontents

\newpage


\section{Introduction}
% Alex
% 0.5 page
Ce rapport présente les résultats de recherches pour l'amélioration de l'intelligence de joueur informatisé(JI).
Les JI qui évoluent entouré de joueur humain font face à un problème critique. En fait ils ne possède pas un intelligence suffisante afin de s'adapter aux multiples situations imprévisible des humains. De plus, les JI sont trop prévisible et rendent les jeux rapidement monotone. À ce jour, l'amélioration de JI l'élément le plus crutiale pour une évolution saine de la compagnie.

Pour répondre à ce problème, l'ingénieur en chef a déterminé l'utilisation de méthodes d'intelligence artificiel. Les recherches effectué comporte plusieurs volets afin de s'assurer de répondre au besoin réelles de la majorités de nos jeux à plateforme multijoueurs. Voici les études qui ont été réalisées:

\textbf{Méthode de recherches}
\begin{enumerate}
 \item Non informé
 \item Informé avec heuristique
\end{enumerate}

\textbf{Méthode de planification}
\begin{enumerate}
 \item recherche d'espaces d'états
 \item partiellement ordonnée
\end{enumerate}

Avec une représentation de l'intelligence artificiel en logique de première ordre, il devient très facile de réutiliser les algorithmes d'IA pour une multitudes de jeux différents. Avec cette généralisation possible, les recherches on été effectué et testés dans l'environnement de jeux BlockBlitz. Cette particularité rend la recherche sur les méthodes de l'intelligence artificiel intéressente afin de réduire le coût de développement de tous les jeux dans la compagnie. L'implémentation des algorithmes ont tous été réaliser à l'aide du language prolog afin d'accélérer le développement. Ce dernier est un langage qui est entièrement basé sur la logique du premier ordre.

Il existe en fait plusieurs autre techniques d'intelligences artificielles que celles présentées précédemments. C'est pourquoi qu en plus de ces recherches, une analyse sommaire de méthodes plus évolué est aussi présenté. À la fin de ce rapport se trouves toutes les recommandations à utiliser pour résouble l'incapacité actuel de JI.


\section{Recommandation et justification de la technique de planification}
% Erick
% 3 pages
L'analyse des différentes techniques de planification a permis de faire ressortir les éléments suivants.  Cependant, étant donné que le choix de la technique de planification est indépendant du choix de la technique de recherche qui l'implémente, ces deux sujets seront traités séparément.

\subsection{Techniques de planification}
Dans les deux cas,il s'agit de trouver une séquence d'action qui permettent d'atteindre un but. Dans le cas de la planification de la recherche par espace d'état, la recherche s'effectue sur l'état de l'environnement de jeu alors que dans le cas de la planification partiellement ordonnée, la recherche s'effectue sur l'environnement d'actions, c'est-à-dire les actions et leur dépendances mutuelles. 

\subsubsection{Planification par recherche d'espace d'état}



\subsubsection{Planification partiellement ordonnée}

\subsection{Techniques de recherche}

\section{Resultats de tests}
% Alex et Erick
% 3 pages

% Criteres
% - Temps de calcul
% - Utilisation memoire
% - Qualite du plan
% - Complexite de developement

% - Couvrir les algorithmes de recherche informee et non-informee

\section{Representation STRIPS ou ADL des actions possibles de BlockBlitz}
% Erick 
% 1.5 pages
Par soucis de concision et de lisibilité, l'environnement de jeu initial et l'environnement de jeu final seront omis des paramètres, des préconditions et des effets des actions présentées bien qu'ils soient utilisés de fa\c con explicite dans le programme remis.  Pour l'ensemble des actions présentées, nous assumerons un environnement de jeu composé des faits suivants, les multiplicités présentées représentent le nombre de faits distincts du type présenté pouvant être retrouvés dans l'environnement, ex: nbJoueur(N), 1 signifie qu'un seul fait de type nbJoueur est présent dans l'environnement et block(...), 1+ signifie que chaque bloc est représenté par son propre fait et qu'il y en a autant que de blocs.

\begin{center}
  \begin{tabular}{@{} ccc @{}}
    \hline
    Fait & Multiplicité & Signification \\ 
    \hline
    NbJoueur(n) & 1 & $n$ est le nombre de joueurs \\ 
    NbBlock(n) & 1 & $n$ est le nombre de blocks \\ 
    NbColonnes(n) & 1 & $n$ est le nombre de colonnes \\ 
    NbRangees(n) & 1 & $n$ est le nombre de rangées \\ 
    Block(id,x,y) & 1+ & $id$ est la valeur du block, \\
                                  && $x$ est la position en selon $x$ \\
                                  && et $y$ est la position selon $y$ \\ 
    Player(id,nom,x,y,blockId) & 1+ & $id$ est l'identifiant unique du joueur \\
                                                              && $nom$ est le nom assigné au joueur \\ 
                                                              && $x$ est la position selon $x$ \\
                                                              && $y$ est la position selon $y$ \\
                                                              && $blockId$ est l'identifiant du bloc possédé par le joueur\\
    Nom(nom) & 1 & $nom$ est le nom du joueur \\ 
    \hline
  \end{tabular}
\end{center}

En plus de ces faits de base, les prédicats suivants sont utilisés pour faciliter l'expression des préconditions et des effets. \\
\\
Il est nécessaire de traduire la direction en coordonnées. Pour ce faire, le prédicat $Direction(posx,posy,newx,newy,direction)$ est utilisé pour déterminer la nouvelle position du joueur à partir de sa position précédente.  Il y a huit entrées pour chacune des direction.  L'entrée pour la direction $1$ est la suivante, les autres entrées sont laissées comme exercice au lecteur.
\begin{equation*}
	Direction(posx,posy,newx,newy, 1) \Leftarrow newy = posy + 1 \wedge newx = posx 
\end{equation*}

Les deux derniers prédicats sont les suivants.  $Empty(posx,posy)$ assure que la case visée ne contient rien, c'est-à-dire aucun joueur ou bloc, et $Exist(posx,posy)$ valide que la case visée appartient bel et bien à l'environnement, c'est-à-dire qu'elle est contenue dans ses limites.
\begin{eqnarray*}
	Empty(posx,posy) \Leftarrow & \lnot Block(\_,posx,posy)  \wedge \lnot Player(\_,\_,posx,posy,\_) \\
	Exist(posx,posy) \Leftarrow & posx  \geq 0 \wedge posy \geq 0 \\
	                                                    \wedge & NbColonnes(nbColonnes)  \wedge NbRangees(nbRangees) \\
	                                                   \wedge & posx < nbColonnes \wedge posy < nbRangees 
\end{eqnarray*}

Il est alors très aisé d'exprimer les préconditions et les effets de chacune des actions.

\begin{tabbing}
\it{Action}\=\it{(Move(direction))}\=,\\
\> PRECOND: \> $Nom(nom) \wedge Player(id,nom,posx,posy,blockid)$ \\ 
\> \> $  \wedge \ Direction(posx,posy,newx,newy,direction) $ \\
\> \> $   \wedge \ Exist(newx,newy) \wedge Empty(newx,newy) $ \\               
\> EFFECT: \>$ \lnot Player(id,nom,posx,posy,blockid) $ \\
\>  \> $\wedge\ Player(id,nom,newx,newy,blockid)$ .\\
\end{tabbing}

Pour le cas où le joueur ne possède pas de bloc, l'action $take$ est la suivante:

\begin{tabbing}
\it{Action}\=\it{(Take(direction))}\=,\\
\> PRECOND: \> $Nom(nom) \wedge Player(id,nom,posx,posy,0)$ \\ 
\> \> $  \wedge \ Direction(posx,posy,newx,newy,direction) $ \\
\> \> $   \wedge \ Exist(newx,newy) \wedge Block(newblockid,newx,newy) $ \\               
\> EFFECT: \>$ \lnot Player(id,nom,posx,posy,0) $ \\
\>  \> $\wedge\ Player(id,nom,newx,newy,newblockid)$ \\
\> \> $\wedge\ \lnot Block(newblockid,newx,newy) $ .\\
\end{tabbing}

Pour le cas où le joueur possède déjà un bloc, l'action $take$ est plutôt:
\begin{tabbing}
\it{Action}\=\it{(Take(direction))}\=,\\
\> PRECOND: \> $Nom(nom) \wedge Player(id,nom,posx,posy,blockid)$ \\ 
\> \> $  \wedge \ Direction(posx,posy,newx,newy,direction) $ \\
\> \> $   \wedge \ Exist(newx,newy) \wedge Block(newblockid,newx,newy) $ \\               
\> EFFECT: \>$ \lnot Player(id,nom,posx,posy,blockid) $ \\
\>  \> $\wedge\ Player(id,nom,newx,newy,newblockid)$ \\
\> \> $\wedge\ \lnot Block(newblockid,newx,newy) $ \\
\> \> $\wedge\ Block(blockid,newx,newy) $ .\\
\end{tabbing}

\begin{tabbing}
\it{Action}\=\it{(Drop(direction))}\=,\\
\> PRECOND: \> $Nom(nom) \wedge Player(id,nom,posx,posy,blockid)$ \\ 
\> \> $  \wedge \  \lnot Player(id,nom,posx,posy,0) $ \\
\> \> $  \wedge \ Direction(posx,posy,newx,newy,direction) $ \\
\> \> $   \wedge \ Exist(newx,newy) \wedge Empty(newx,newy) $ \\               
\> EFFECT: \>$ \lnot Player(id,nom,posx,posy,blockid) $ \\
\>  \> $\wedge\ Player(id,nom,posx,posy,0)$ \\
\> \> $\wedge\ Block(blockid,newx,newy) $ .\\
\end{tabbing}

Bien que la problématique présente deux cas distincts pour l'attaque, avec la représentation proposée, la même description s'applique dans les deux cas (joueur possède ou non un bloc).  De plus, pour fins de planification, nous allons considérer que l'action réussi toujours.:

\begin{tabbing}
\it{Action}\=\it{(Attack(direction))}\=,\\
\> PRECOND: \> $Nom(nom) \wedge Player(id,nom,posx,posy,blockid)$ \\ 
\> \> $  \wedge \ Direction(posx,posy,newx,newy,direction) $ \\
\> \> $  \wedge \ Player(id2,nom2,newx,newy,blockid2) $ \\
\> \> $  \wedge \ \lnot Player(id2,nom2,newx,newy,0) $ \\
\> \> $ \wedge\ \lnot Block(bid,newx,newy) \wedge \ Exist(newx,newy)$ \\               
\> EFFECT: \>$ \lnot Player(id,nom,posx,posy,blockid) $ \\
\>  \> $\wedge\ Player(id,nom,posx,posy,blockid2)$ \\
\>  \> $\wedge\ \lnot Player(id2,nom2,newx,newy,blockid2)$ \\
\>  \> $\wedge\ Player(id2,nom2,newx,newy,blockid)$ .\\
\end{tabbing}

Finalement, l'action $None$ ne possède aucune précondition et n'a aucun effet sur l'environnement.

\section{Description et une explication du vocabulaire des parties principales de la solution}

\section{Description de parties principales de notre solution}
% Alex
% 1 page

\section{Recommandation de la pertinence des methodes d'algorithme évolué}
% Alex
% 1 page

\section{Conclusion}
% Alex
% 0.5 page

\end{document}