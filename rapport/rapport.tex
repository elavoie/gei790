\documentclass[12pt,english,frenchb,letterpaper]{article}
\usepackage[T1]{fontenc}
\usepackage[utf8]{inputenc}
\usepackage{geometry}
\usepackage{setspace}
\usepackage{graphics}
\usepackage{graphicx}
\usepackage{amsmath,amssymb,amsfonts,textcomp}
\usepackage{color}
\usepackage{calc}
\usepackage{verbatim}
\usepackage{longtable}
\geometry{verbose, letterpaper,tmargin=3cm,bmargin=3cm,lmargin=3cm,rmargin=3cm}
\usepackage[french]{babel}

\begin{document}
\thispagestyle{empty}
\begin{center}
{\large\em GEI790:  Intelligence Artificielle formelle}
\vfill
Rapport de l'App 1\\
présenté professorale de S8\\
\vfill
par:\\
Alexandre Malo\hspace{2cm} 05 659 076\\
Erick Lavoie\hspace{2cm} 05 646 456\\
\vfill
Date: \today
\end{center}

\newpage
\onehalfspacing


\tableofcontents

\newpage


\section{Introduction}
% Alex
% 0.5 page
Ce rapport présente les résultats de recherches pour l'amélioration de l'intelligence de joueur informatisé(JI).
Les JI qui évoluent entouré de joueur humain font face à un problème critique. En fait ils ne possède pas un intelligence suffisante afin de s'adapter aux multiples situations imprévisible des humains. De plus, les JI sont trop prévisible et rendent les jeux rapidement monotone. À ce jour, l'amélioration de JI l'élément le plus crutiale pour une évolution saine de la compagnie.

Pour répondre à ce problème, l'ingénieur en chef a déterminé l'utilisation de méthodes d'intelligence artificiel. Les recherches effectué comporte plusieurs volets afin de s'assurer de répondre au besoin réelles de la majorités de nos jeux à plateforme multijoueurs. Voici les études qui ont été réalisées:

\textbf{Méthode de recherches}
\begin{enumerate}
 \item Non informé
 \item Informé avec heuristique
\end{enumerate}

\textbf{Méthode de planification}
\begin{enumerate}
 \item recherche d'espaces d'états
 \item partiellement ordonnée
\end{enumerate}

Avec une représentation de l'intelligence artificiel en logique de première ordre, il devient très facile de réutiliser les algorithmes d'IA pour une multitudes de jeux différents. Avec cette généralisation possible, les recherches on été effectué et testés dans l'environnement de jeux BlockBlitz. Cette particularité rend la recherche sur les méthodes de l'intelligence artificiel intéressente afin de réduire le coût de développement de tous les jeux dans la compagnie. L'implémentation des algorithmes ont tous été réaliser à l'aide du language prolog afin d'accélérer le développement. Ce dernier est un langage qui est entièrement basé sur la logique du premier ordre.

Il existe en fait plusieurs autre techniques d'intelligences artificielles que celles présentées précédemments. C'est pourquoi qu en plus de ces recherches, une analyse sommaire de méthodes plus évolué est aussi présenté. À la fin de ce rapport se trouves toutes les recommandations à utiliser pour résouble l'incapacité actuel de JI.


\section{Recommandation et justification de la technique de planification}
% Erick
% 3 pages

\section{Resultats de tests}
% Alex et Erick
% 3 pages

% Criteres
% - Temps de calcul
% - Utilisation memoire
% - Qualite du plan
% - Complexite de developement

\section{Representation STRIPS ou ADL des actions possibles de BlockBlitz}
% Erick 
% 1.5 pages

\section{Description du vocabulaire de nos parties principales }
% En logique du premier ordre
% Erick 
% 1 page

\section{Description de parties principales de notre solution}
% Alex
% 1 page

\section{Recommandation de la pertinence des methodes d'algorithme évolué}
% Alex
% 1 page

\section{Conclusion}
% Alex
% 0.5 page

\end{document}